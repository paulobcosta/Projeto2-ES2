	\documentclass[12pt,a4paper]{article}
	\usepackage[utf8]{inputenc}
	\usepackage[portuguese]{babel}
	\usepackage[T1]{fontenc}
	\usepackage{amsmath}
	\usepackage{amsfonts}
	\usepackage{amssymb}
	\usepackage{graphicx}
	\usepackage[left=2.5cm,right=2.5cm,top=2.5cm,bottom=2.5cm]{geometry}
	\author{Paulo Batista da Costa}
	\title{Laudo PBR refente ao projeto II }
	
	\begin{document}
	
        \begin{titlepage}
        \LARGE
        	\begin{center}
        	\vspace{5cm} 
        	\textbf{Universidade Tecnológica Federal do Paraná \\ \vspace{1.8cm}}
        	\includegraphics[scale=0.35]{logoutfpr.jpg} \\ \vspace{1.8cm}
        	\textit{Engenharia de Software II} \vspace{2cm} \\
        	Paulo Batista da Costa \\ 1509764 \vspace{2cm} \\ 
        	PROJETO 2 - PLANEJAMENTO DE DESENVOLVIMENTO DE SOFTWARE \vspace{2cm} \\
        	Departamento de Ciência da Computação (DACOM) 
        	
        	\end{center}
        \end{titlepage}	
	
		\tableofcontents
		\newpage
		\section{Da Natureza Deste Documento}
		\paragraph{} Este documento é referente ao planejamento de atividades de desenvolvimento de software do projeto 2. Tal projeto é proposto como base de atividades na disciplina de engenharia de software II. Desse modo, o objetivo deste documento é explicitar o conjunto de atividades de desenvolvimento do projeto 2 de forma a dividi-lo em conjunto de tarefas ou grupo de atividades que serão executadas. 
		
		\paragraph{} Por sua vez, o grupo de desenvolvimento seguirá este plano como base fundamental para caminhar o desenvolvimento. Aqui também é analisado uma estimativa de tempo para cada parte do processo de elaboração de atividades (compatível com a estimativa de tempo realizada previamente). Apenas a critério de informação, o desenvolvimento das atividades darão início  a partir do momento em que este documento estiver lançado no repositório de desenvolvimento.
		\section{Descrição de Atividades}
		\paragraph{} A primeira etapa do planejamento de software é a divisão do desenvolvimento em etapas. Assim, o conjunto tem partes distinguíveis entre si e estas são distribuídas de maneira sensata aos colaboradores do projeto. Cada etapa de desenvolvimento será de responsabilidade de um integrante do projeto. Na subseção a seguir estão explícitas as etapas e as respectivas distribuições de tarefas.
		\subsection{Etapa 1: Diagramação}
		\subsubsection{Diagramação: Casos de Uso}
		\subsubsection{Diagramação: Diagramas de Classe}
		\subsection{Etapa 2: Implementação}
		\subsubsection{Implementação: Classes}
		\subsubsection{Implementação: Banco de Dados}
		\subsubsection{Implementação: Interface Gráfica}
		\subsection{Etapa 3: Realização de testes}
		\subsubsection{Testes: Execução de teste de lógica programada}
		\subsubsection{Testes: Execução de teste em Banco de Dados}
		\subsubsection{Testes: Execução de teste voltado para interface gráfica}
		\subsection{Etapa 4: Correções de \textit{bugs} e outros defeitos}
		\subsubsection{Correção: Correção de defeitos de lógica de programação}
		\subsubsection{Correção: Correção de defeitos em Banco de Dados}
		\subsubsection{Correção: Correção de defeitos e imperfeições de interface gráfica}
		\subsection{Etapa 5: Validação de Correções}
		\subsubsection{Vaçidação: validação das correções de lógica de programação}
		\subsubsection{Validação: validação das correções de Banco de Dados}
		\subsubsection{Validação: validação das correções de imperfeições de interface gráfica}
		\end{document}