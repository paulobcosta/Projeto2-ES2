\documentclass[12pt,a4paper,final]{report}
\usepackage[utf8]{inputenc}
\usepackage{amsmath}
\usepackage{amsfonts}
\usepackage{amssymb}
\usepackage{graphicx}
\usepackage[portuguese]{babel}
\author{Gerente: Paulo Batista da Costa}
\title{Laudo - Projeto 2}
\begin{document}
\maketitle
\section*{Análise do Documento}
\begin{itemize}
\item Os elementos e os dados foram bem definidos, exceto os dados relativos a login (ficou mal definido) -- não define como o login deve ser efetuado e critérios de segurança.
\item As interfaces que são especificadas são consistentes, entretanto nem todas foram especificadas dando margem a contradições interpretativas
\item Foi possível definir todos os tipos de dados.
\item Nem todas as informações necessárias estão disponíveis. Informações como interfaces mal descritas e ações não especificadas por parte de usuário.
\item Tudo o que está na documentação faz sentido, porém nem todas as informações necessárias estão contidas no documento.


\item Não foi possível elaborar casos de testes para cada requisito (testes unitários, de forma particular). Entretanto os testes automaticamente abrangeram vários requisitos funcionais por vez.

\item Não foi possível ter certeza se os testes gerariam as saídas esperadas (corretas) -- Isso é possível saber após a implementação.

\item Não existe uma forma de interpretação que o programador possa estar se baseando de forma diferente do testador, pois os requisitos estão descritos de forma simples. Mas o domínio dos dados não está bem definido. O usuário não está devidamente definido (o que é um usuário pro sistema ) -- define apenas as ações do usuário. 

\item Os testes foram elaborados para serem funcionais para todo o sistema. Os testes seriam basicamente para CRUD (verificando em suma operações de banco de dados). Assim, testes similares resultariam em resultados iguais. Mais uma vez, o defeito é a não definição do domínio de dados (falta especificação).

\item A especificação fez sentido (em consideração aos conhecimentos do testador), basicamente é a simulação de um banco, um banco já tem interface para as ações desse tipo de sistema. 
\end{itemize}


\end{document}