\documentclass[12pt,a4paper,final]{report}
\usepackage[utf8]{inputenc}
\usepackage{amsmath}
\usepackage{amsfonts}
\usepackage{amssymb}
\usepackage{graphicx}
\usepackage[portuguese]{babel}
\author{Gerente: Paulo Batista da Costa}
\title{Laudo - Projeto 2}
\begin{document}
\maketitle
\section*{Introdução}
\begin{quotation}
	Este documento remete ao laudo realizado através da análise do processo de PBR que visou identificar possíveis defeitos na documentação fornecida. Este documento é referente ao projeto 2. Aqui é possível verificar defeitos e considerações gerais, bem como o controle de horas consumidas e as suas respectivas estimativas.
\end{quotation}
\section*{Análise da Documentação - resultado do PBR}
\subsection*{Defeitos}
\begin{enumerate}
	\item Os elementos e os dados foram bem definidos, exceto os dados relativos a login (ficou mal definido) -- não define como o login deve ser efetuado e critérios de segurança.
	\item As interfaces que são especificadas são consistentes, entretanto há itens não especificados que que remete a uma margem para contradições interpretativas.
	\item Nem todas as informações necessárias estão disponíveis. Informações como interfaces mal descritas e ações não especificadas por parte de usuário.
	\item Tudo o que está na documentação faz sentido, porém nem todas as informações necessárias estão contidas no documento.
	\item Não foi possível elaborar casos de testes para cada requisito (testes unitários, de forma particular). Entretanto os testes automaticamente abrangeram vários requisitos funcionais por vez.
	\item Não foi possível ter certeza se os testes gerariam as saídas esperadas (corretas) -- Isso é possível saber após a implementação.
\end{enumerate}
\subsection*{Considerações gerais}
\begin{enumerate}


\item Foi possível definir todos os tipos de dados.








\item Não existe uma forma de interpretação que o programador possa estar se baseando de forma diferente do testador, pois os requisitos estão descritos de forma simples. Mas o domínio dos dados não está bem definido. O usuário não está devidamente definido (o que é um usuário pro sistema ) -- define apenas as ações do usuário. 

\item Os testes foram elaborados para serem funcionais para todo o sistema. Os testes seriam basicamente para CRUD (verificando em suma operações de banco de dados). Assim, testes similares resultariam em resultados iguais. Mais uma vez, o defeito é a não definição do domínio de dados (falta especificação).

\item A especificação fez sentido (em consideração aos conhecimentos do testador), basicamente é a simulação de um banco, um banco já tem interface para as ações desse tipo de sistema. 
\end{enumerate}
\newpage
\section*{Relatório - Controle de Esforço}
\vspace{0.75cm}
\begin{table*}[ht]
	 \centering
\begin{tabular}{||c|c|c||}\hline
	\textbf{Visão} & \textbf{Tempo Estimado} & \textbf{Tempo Consumido} \\ \hline
	Projetista(1) & 20 Horas & 4 Horas e 20 min \\ \hline
	Testador(2) & 8 Horas & 1 Hora e 15 min \\ \hline
	Usuário(3) & 4 Horas & 55 min \\ \hline
\end{tabular} 
\caption{Tabela de Controle de esforço}
\end{table*}
\vspace{0.75cm}
\begin{enumerate}
	
\item Projetista: 

	
	\textbullet(29/03/2016) -- 13h e 50min  - 15h e 30min \\
	\textbullet(30/03/2016) -- 15h e 50min  - 17h e 30min \\
	\textbullet(31/03/2016) -- 9h e 30min  - 10h e 30min 

  

\item Testador:

	\textbullet(31/03/2016) -- 12h e 15min - 13h e 30min

\item Usuário: 

	\textbullet(06/04/2016) -- 0h e 30min - 1h e 25min
 
\end{enumerate}
\vspace{0.75cm}
\begin{quotation}
	\checkmark É notável que as estimativas excederam os tempos que de fato foram consumidos pelos membros da equipe. Tal erro é justificado pela inexperiência relativa à prática do PBR.
\end{quotation}
\end{document}