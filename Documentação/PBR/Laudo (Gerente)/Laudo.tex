	\documentclass[12pt,a4paper]{article}
	\usepackage[utf8]{inputenc}
	\usepackage[portuguese]{babel}
	\usepackage[T1]{fontenc}
	\usepackage{amsmath}
	\usepackage{amsfonts}
	\usepackage{amssymb}
	\usepackage{graphicx}
	\usepackage[left=2.5cm,right=2.5cm,top=2.5cm,bottom=2.5cm]{geometry}
	\author{Paulo Batista da Costa}
	\title{Laudo PBR refente ao projeto II }
	
	\begin{document}
	
        \begin{titlepage}
        \LARGE
        	\begin{center}
        	\vspace{5cm} 
        	\textbf{Universidade Tecnológica Federal do Paraná \\ \vspace{1.8cm}}
        	\includegraphics[scale=0.35]{logoutfpr.jpg} \\ \vspace{1.8cm}
        	\textit{Engenharia de Software II} \vspace{2cm} \\
        	Paulo Batista da Costa \\ 1509764 \vspace{2cm} \\ 
        	PROJETO 2 - LAUDO PBR \vspace{2cm} \\
        	Departamento de Ciência da Computação (DACOM) 
        	
        	\end{center}
        \end{titlepage}	
	
		\tableofcontents
		\newpage
		\section{Da Natureza deste Documento} 
		\paragraph{} 
		Este documento é referente ao laudo realizado através da análise do processo de \textbf{PBR} que visou identificar possíveis defeitos na documentação fornecida. Este é específico ao laudo gerencial do projeto 2. Aqui é possível verificar defeitos e considerações gerais, bem como o controle de horas consumidas e as suas respectivas estimativas.
		
		\section{Resultado}
		\subsection{Análise Documental: defeitos encontrados}
		\begin{enumerate}
		\item Os elementos e os dados foram bem definidos, exceto os dados relativos a login (mal definido na documentação) -- não define como o login deve ser efetuado e critérios de segurança.
		\item As interfaces especificadas são consistentes, entretanto há elementos não especificados promovem margens para contradições interpretativas.
		\item Nem todas as informações necessárias estão disponíveis, há algumas interfaces mal descritas e ações não especificadas por parte de usuário.
		\item Tudo o que está na documentação faz sentido, porém nem todas as informações necessárias estão contidas na especificação.
		\item Não foi possível elaborar casos de testes para cada requisito (testes unitários, de forma particular). Entretanto os testes automaticamente abrangeram vários requisitos funcionais concomitantemente.
		\item Não foi possível obter certeza se os testes geraram as saídas esperadas (corretas) -- Isso é possível saber após a implementação.
	\end{enumerate}
	\subsection{Condições Gerais}
		\begin{enumerate}
		\item Foi possível definir todos os tipos de dados.
	
	
	
	
	
	
	
	
	\item Não existe uma forma de interpretação que o programador possa estar se baseando de forma diferente do testador, pois os requisitos estão descritos de forma simples. Mas o domínio dos dados não está bem definido. O usuário não está devidamente definido (o que é um usuário pro sistema ) -- define apenas as ações do usuário. 
	
	\item Os testes foram elaborados para serem funcionais para todo o sistema. Os testes seriam basicamente para CRUD (verificando em suma operações de banco de dados). Assim, testes similares resultariam em resultados iguais. Mais uma vez, o defeito é a não definição do domínio de dados (falta especificação).
	
	\item A especificação fez sentido (em consideração aos conhecimentos do testador), basicamente é a simulação de um banco, um banco já tem interface para as ações desse tipo de sistema. 
	\end{enumerate}
		\newpage
	\section{Controle de esforço}
	\paragraph{} Nesta seção há a explicitação a respeito do tempo consumido pelos integrantes do projeto na elaboração do \textbf{PBR}. O tempo foi fornecido por cada participante, cabendo à gerência a confiabilidade nos membros da equipe quanto a veracidade da informação relativa ao esforço consumido na etapa de busca de defeitos via \textbf{PBF}.
	\vspace{0.45cm}
		\begin{table*}[ht]
		 \centering
	\begin{tabular}{||c|c|c||}\hline
		\textbf{Visão} & \textbf{Tempo Estimado} & \textbf{Tempo Consumido} \\ \hline
		Projetista(1) & 20 Horas & 4 Horas e 20 min \\ \hline
		Testador(2) & 8 Horas & 1 Hora e 15 min \\ \hline
		Usuário(3) & 4 Horas & 55 min \\ \hline
	\end{tabular} 
	\caption{Tabela de Controle de esforço}
	\end{table*}
		
	\end{document}